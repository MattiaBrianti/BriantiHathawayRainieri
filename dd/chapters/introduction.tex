% !TeX root = ../dd.tex
\section{Purpose}
The purpose of this document is to present a technical description of the S\&C platform. It is intended for developers that have to implement requirements and can serve as a contractual agreement between the customer and the contractors. Additionally, this document aims to provide the customer with a clear and precise explanation of the system's functionalities and constraints.
\section{Scope}
The Students\&Companies platform is a tool that allows students and companies to facilitate their research for internships and interns. It offers students the possibility to join an internship project and companies to share the projects they are offering. 
For this software system a 3-tier architecture is the best choice, since there are three main levels such as a presentation level for users, an application tier to manage the logic of the internships, the request and communication and a data layer that copes with databases.

\section{Definitions, Acronyms, Abbreviations}

\subsection{Definitions}
\begin{itemize}
    \item \textbf{OAuth Access Token}: is a temporary credential that allows a client to access protected resources on behalf of a user or an app.
    \item \textbf{OAuth Refresh Token}: is a credential used to obtain a new access token without re-authenticating.
\end{itemize}


\subsection{Acronyms}
\begin{description}[leftmargin=0pt] 
    \item [SSO:] Single Sign On
    \item [API:] Application Programming Interface
    \item [CV:] Curriculum Vitae
    \item [S\&C:] Students\&Companies
    \item [IDE:] Integrated Development Environment
    \item[RASD:] Requirements Analysis \& Specification Document
\end{description}


\subsection{Abbreviations}
\begin{description}[leftmargin=0pt]
    \item[e.g.:] For example
    \item [w.r.t.:] With reference to
    \item [ID: ] Identifier
\end{description}

\section{Revision history}

\begin{itemize}
    \item \textbf{1.0} (07th January 2025) - Initial release
\end{itemize}

\section{Reference Documents}

\begin{description}[leftmargin=0pt]
    \item[Specification document:] \emph{"Assignment RDD AY 2024-2025"}
    \item[UML official specification:] \url{https://www.omg.org/spec/UML/}
    \item[Requirements Analysis and Specification Document:] \emph{"RASD"}
\end{description}


\section{Document Structure}

\begin{enumerate}
    \item \textbf{Section 1: Introduction} \\
    This section gives a brief description of the project. It analyzes summarily the goals and the purpose of the platform described in the RASD. It also contains lists of definitions, acronyms and abbreviations used in the document.  
    \item \textbf{Section 2: Architectural Design} \\
    This section firstly contains an high level description of the components and their interactions. It uses diagrams to describe the architecture of the system. Also, at the end of the section, it describes design choices, styles, patterns and paradigms.
    \item \textbf{Section 3: User Interface Design} \\
    This section provides an overview on how the UI of the system will look like.
    \item \textbf{Section 4: Requirements Traceability} \\
    This section maps the requirements that have been defined in the RASD to the design elements defined in this document.
    \item \textbf{Section 5: Implementation, Integration and Test Plan} \\
    This section shows the order in which the subcomponents of the system will be implemented as well as the order in which subcomponents will be integrated and how to test the integration.
    \item \textbf{Section 6: Effort Spent} In the sixth section are included information about the number of hours each group member has worked for this document.
\end{enumerate}