% !TeX root = ../dd.tex
\section{Scope}
The Students\&Companies platform is a platform that allows students and companies to facilitate their research for internships and interns. It offers students the possibility to join an internship project and companies to share the projects they are offering. 
For this software system a 3-tier architecture is the best choice.  

\section{Definitions, Acronyms, Abbreviations}

\subsection{Definitions}
\begin{description}[leftmargin=0pt]
    \item[Slug:] a human-readable and URL-friendly (as in, limited to lowercase ASCII characters) string which
          identifies a particular resource. It is typically used to identify resources in URLs, but can also be used
          as IDs instead of a more traditional way such as integers
    \item[GitHub Repository Slug:] a slug with the structure `<repository owner>/<repository name>' which uniquely
          identifies a repository across all of the ones hosted on GitHub. Can typically be seen in a repository URL
\end{description}

\subsection{Acronyms}
\begin{description}[leftmargin=0pt]
    \item[CKB:] CodeKataBattle
    \item[API:] Application Programming Interface
    \item[SAT:] Static Analysis tools
    \item[GH:] GitHub
    \item[SSO:] Single Sign On
    \item[UUID:] Universal Unique Identifier
    \item[DB:] DataBase
    \item[DBMS:] DataBase Management System
    \item[RPC:] Remote Procedure Call
    \item[REST:] REpresentational State Transfer
    \item[SPA:] Single Page App
    \item[CDN:] Content Delivery Network
\end{description}

\subsection{Abbreviations}
\begin{description}[leftmargin=0pt]
    \item[e.g.:] For example
    \item[repo:] Repository
    \item[ID:] Identifier
\end{description}

\section{Revision history}

\begin{itemize}
    \item \textbf{1.0} (7th January 2024) {-} Initial release
\end{itemize}

\section{Reference Documents}

\begin{description}[leftmargin=0pt]
    \item[Specification document:] \emph{"Assignment RDD AY 2023-2024"}
    \item[UML official specification:] \url{https://www.omg.org/spec/UML/}
    \item[Requirements Analysis and Specification Document:] \emph{"RASD"}
\end{description}


\section{Document Structure}

\begin{enumerate}
    \item \textbf{Section 1: Introduction} \\
    This section gives a brief description of the problem and the scope of the system.
    It also contains the list of definitions, acronyms and abbreviations that might be encountered
    while reading the document.
    Additionally, there's the revision history of the document, which keeps track of the various
    version, their release date and their changes.
    \item \textbf{Section 2: Architectural Design} \\
    This section is about the architecture of the system. 
    It firstly gives a high level overview of the architectural choices.
    Then it presents in details the components and deployment views, including server, client and data components (DB schemas).
    Additionally, it contains sequence diagrams that represent the runtime view of the system, as well as the component interfaces.
    Finally, this section focus on design choices, styles, patterns and paradigms
    \item \textbf{Section 3: User Interface Design} \\
    This section provides an overview on how the UI of the system will look like.
    \item \textbf{Section 4: Requirements Traceability} \\
    This section maps the requirements that have been defined in the RASD to the design elements defined in this document.
    \item \textbf{Section 4: Implementation, Integration and Test Plan} \\
    This section shows the order in which the subcomponents of the system will be implemented as well
    as the order in which subcomponents will be integrated and how to test the integration.
\end{enumerate}